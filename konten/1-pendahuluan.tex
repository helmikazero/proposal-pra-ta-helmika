\section{PENDAHULUAN}
\label{chap:pendahuluan}

% Ubah bagian-bagian berikut dengan isi dari pendahuluan

Dalam bab ini akan dibahas latar belakang, permasalahan, tujuan, serta batasan masalah pada penelitian ini.

\subsection{Latar Belakang}
\label{sec:latarbelakang}

Keselamatan dan Kesehatan Kerja atau biasa disingkat K3 menjadi usaha untuk meningkatkan kualitas lingkungan kerja agar menjadi lebih aman dan sehat bagi segala pihak yang ada di lingkungan tersebut. Aman dan sehat bisa dalam artian bebas dari kecelakaan, kebakaran, ledakan, lingkungan tercemar, atau wabah penyakit. Tentu saja hal - hal tersebut perlu dihindari karena dapat memberikan dampak kerugian material dan bahkan melayangnya nyawa manusia. Aturan K3 sendiri diatur dalam bentuk norma oleh regulasi pemerintah Republik Indonesia lewat UUD 1945 pasal 27 ayat 2 tentang filosofi penghidupan yang layak, UU No 1 tahun 1970 tentang keselamatan kerja, Undang-Undang No. 13 Tahun 2003 pasal 86 dan 87 Kewajiban penerapan Sistem Manajemen Keselamatan dan Kesehatan Kerja (SMK3), Peraturan Pemerintah No.50 Tahun 2012 tentang Sistem Manajemen Keselamatan dan Kesehatan Kerja (SMK3), dan peraturan pelaksanaan lainnya dari Permenaker, Instruksi Menteri, Pekmenaker.  \cite{ahlik3umum-k3indonesia_2021}
Medan dari proyek konstruksi dapat dianggap sebagai lingkungan yang penuh dengan resiko menjadikannya suatu hal yang patut dipertimbangkan. Bukan hal yang jarang dimana para personel lapangan yang bekerja di suatu proyek konstruksi mengalami cedera akibat hal - hal tertentu. Mulai dari debu konstruksi, puing - puing berterbangan, jatuh dari ketinggian, atau tertimpa benda. Cedera kepala oleh jatuh ketinggian atau tertimpa benda dapat berakibat fatal pada nyawa personil lapangan. \cite{li2020deep}
Helm keselamatan kerja atau \emph{Hardhat} dalam Bahasa inggris atau juga kadang disebut Helm proyek merupakan salah satu bentuk APD K3 yang berfungsi untuk melindungi kepala pengguna dari benturan. Bentuk benturan contohnya kejatuhan benda tajam atau berat yang sekiranya jika tidak menggunakan pelindung akan berdampak fatal pada kepala personil proyek. Selain benturan, helm juga digunakan untuk melindungi kepala penggunanya dari percikan api dan berbagai bentuk serpihan terbang yang biasa ada di lokasi kerja.\cite{k3_mutiaramutu}
Adanya aturan penggunaan helm di suatu proyek konstruksi dengan dasar K3 belum tentu menjamin penggunaan efektif dari helm proyek tersebut. 
Berdasarkan Data Kecelakaan dan Penyakit Akibat Kerja Triwulan II 2020 dari Kemnaker, kecelakaan kerja Tipe A yang meliputi "Terbentur pada umumnya menunjukan kontak atau persinggungan dengan benda tajam atau benda keras yang menyebabkan tergores, terpotong, tertusuk dll" mencapai angka 878 kecelakaan dimana menjadi angka terbesar dibanding tipe kecelakaan lain dengan Tipe J (lain-lain) dengan angka 637 dan Tipe C (terjepit) dengan angka 439 \cite{satudata_kecelakaan_kerja}.
Pengawasan terhadap penerapan Keselamatan Kesehatan Kerja pada suatu proyek seperti menggunakan helm Hard Hat secara konvensional sudah sering dilakukan. Personil pengawas yang dikerahkan untuk memastikan para pekerja di suatu proyek mematuhi aturan keselamatan kerja. Misalnya pengawas ditugaskan untuk mengingatkan pekerja proyek yang tidak menggunakan helm proyek dengan tepat atau bahkan tidak digunakan sama sekali.\cite{li2020deep} \cite{masrully2019menakar}

\subsection{Permasalahan}
\label{sec:permasalahan}

Disebutkan pada latar belakang bahwa pengawasan konvensional menggunakan personel manusia untuk menjadi pengawas akan penggunaan helm proyek sebagai APD Keselamatan dan Kesehatan Kerja. Tetapi, teruntuk lokasi proyek konstruksi yang luas dan dengan jumlah pekerja proyek yang banyak menjadikan pengawasan hal yang sulit untuk dilakukan oleh pengawas atau supervisor manusia. Maka dari itu ditarik permasalahan yaitu Helm keselamatan kerja sebagai salah satu APD K3 masih sering diabaikan ditambah dengan pengawasan penggunaan helm sebagai APD K3 yang masih dilakukan secara manual oleh petugas pengawas. 


\subsection{Penelitian Terkait}
Terdapat beberapa penelitian yang sudah dilakukan sebelumnya yang berkaitan dengan penelitian ini yaitu seperti
\emph{Deep Learning Based Safety Helmet Detection in Engineering Management Based on Convolutional Neural Networks} yang dilakukan oleh
Li dan teman teman pada tahun 2020 tentang metode deteksi helm keselamatan kerja secara real time berbasis deep learning pada lokasi konstruksi. Li dan teman – teman menggunakan SSD-MobileNet yang berbasis dari CNN. Menggunakan dataset yang berjumlah 3261 gambar helm keselamatan. SSD- Mobilenet dipilih dibanding R-CNN dengan maksud pendeteksian yang lebih cepat dan cocok untuk real – time walau tidak seakurat R-CNN. \cite{li2020deep}
Lalu Deteksi Penggunaan Helm Pada Pengendara Bermotor Berbasis Deep Learning 
oleh Yusuf Umar pada tahun 2020 yang melakukan penelitian tentang deteksi penggunaan helm pada pengendara bermotor. Pada penelitiannya menggunakan YOLOv3 yang berbasis dari CNN. Pada sistem yang dikembangkan dapat memberikan bounding box ke pengendara lalu dalam bounding box pengendara terdapat boundbox lain dari kepala hingga dada pengendara untuk mendeteksi penggunaan helm motor ada atau tidak. \cite{hanafi2020deteksi}
Dan juga \emph{Safety Helmet Detection Based on YOLOv5} oleh Zhou dan teman - teman pada awal tahun 2021 yang berupa penelitian deteksi helm keselamatan kerja yang berbasis dari YOLOv5. Pada penelitiannya Zhou dan teman - teman melakukan perbandingan dengan 4 model dari YOLOv5 yang meliputi YOLOv5s, YOLOv5m, YOLOv5l, dan YOLOv5. Selain itu Zhou dan teman - teman menggunakan dataset yang berisi 6054 gambar yang dikumpulkan dari
internet dan di anotasi sendiri. Label anotasinya ada dua yaitu "Alarm" yang merupakan kepala tanpa helm dan
"helmet" yang merupakan kepala dengan helm keselamatan kerja\cite{zhou_zhao_nie_2021}.

\subsection{Gap Penelitian}
Pada penelitian \emph{Deep Learning-Based Safety Helmet Detection in Engineering Management Based on Convolutional Neural Networks} oleh Li dan kawan - kawan , pendeteksian hanya mendeteksi helm keselamatannya sendiri tetapi belum mendeteksi personel yang tidak menggunakan helm keselamatan. Selain itu, Li dan kawan - kawan tidak menggunakan YOLO dalam penelitiannya dan memilih menggunakan SSD-MobileNet.
\par Pada penelitian Deteksi Penggunaan Helm Pada Pengendara Bermotor Berbasis Deep Learning oleh Yusuf Umar hanya diimplementasikan pada deteksi helm pada pengendara bermotor dan belum untuk helm proyek. Sekiranya metode yang akan digunakan untuk pengembangan sama tetapi dataset yang digunakan akan berbeda.
\par Pada penelitian \emph{Safety Helmet Detection Based on YOLOv5} oleh Zhou dan teman - teman melakukan deteksi pada pengguna helm keselamatan kerja dan yang tidak. Tetapi dari hasil deteksi ini belum ada tindak lanjut dari hasil output jika
sekiranya terdapat personel yang tidak menggunakan helm keselamatan kerja. 


\subsection{Tujuan Penelitian}
\label{sec:Tujuan}

Dari permasalahan yang disebutkan, dapat dapat ditentukan tujuan yaitu merancang sistem yang dapat mendeteksi penggunaan helm keselamatan kerja secara real-time.
% \subsection{Batasan Masalah}
% \label{sec:batasanmasalah}

% Batasan-batasan dari pengembangan Deteksi Helm Keselmatan Kerja menggunakan CNN meliputi hal - hal berikut:

% \begin{enumerate}[nolistsep]

%   \item Diasumsikan sistem pengawasan diletakkan pada \emph{checkpoint} masuk kawasan konstruksi

%   \item Metode deteksi penggunaan helm keselematan kerja yang akan digunakan pada penelitian ini adalah \emph{You Only Look Once} (YOLO) versi 5 atau YOLOv5.

%   \item Jenis input yang akan digunakan untuk deteksi adalah input dari kamera yang diletakkan pada checkpoint masuk kawasan konstruksi.
%   \item Sistem hanya mendeteksi "kepala menggunakan helm" dan "kepala yang tidak mengenakan helm".

% \end{enumerate}



