\section{HASIL YANG DIHARAPKAN}

Pada akhir pengembangan diharapkan dihasilkan sistem yang dapat membedakan dan mendeteksi personil lapangan yang menggunakan helm dan tidak menggunakan helm secara otomatis dan real-time.

\section{RENCANA KERJA}

% Ubah tabel berikut sesuai dengan isi dari rencana kerja
\newcommand{\w}{}
\newcommand{\G}{\cellcolor{gray}}
\begin{table}[h!]
  \begin{tabular}{|p{3.5cm}|c|c|c|c|c|c|c|c|c|c|c|c|c|c|c|c|}

    \hline
    \multirow{2}{*}{Kegiatan} & \multicolumn{16}{|c|}{Minggu} \\
    \cline{2-17} &
    1 & 2 & 3 & 4 & 5 & 6 & 7 & 8 & 9 & 10 & 11 & 12 & 13 & 14 & 15 & 16 \\
    \hline

    % Gunakan \G untuk mengisi sel dan \w untuk mengosongkan sel
    Studi Literatur &
    \G & \G & \G & \G & \G & \w & \w & \w & \w & \w & \w & \w & \w & \w & \w & \w \\
    \hline

    Pembuatan Model &
    \w & \w & \w & \G & \G & \G & \G & \G & \w & \w & \w & \w & \w & \w & \w & \w \\
    \hline

    Perancangan Sistem &
    \w & \w & \w & \w & \w & \w & \G & \G & \G & \G & \G & \w & \w & \w & \w & \w \\
    \hline

    Implementasi dan Evaluasi &
    \w & \w & \w & \w & \w & \w & \w & \w & \w & \w & \w & \G & \w & \w & \w & \w \\
    \hline
    
    Penulisan Laporan &
    \w & \w & \w & \w & \w & \w & \w & \w & \w & \w & \w & \w & \G & \G & \G & \G \\
    \hline

  \end{tabular}
\end{table}