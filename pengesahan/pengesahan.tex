\begin{flushleft}
  % Ubah kalimat berikut sesuai dengan nama departemen dan fakultas
  \textbf{Departemen Teknik Komputer - FTEIC}\\
  \textbf{Institut Teknologi Sepuluh Nopember}\\
\end{flushleft}


\begin{center}
  % Ubah detail mata kuliah berikut sesuai dengan yang ditentukan oleh departemen
  \underline{\textbf{EC184701 - PRA TUGAS AKHIR (2 SKS)}}
\end{center}

\begin{adjustwidth}{-0.2cm}{}
  \begin{tabular}{lcp{0.7\linewidth}}

    % Ubah kalimat-kalimat berikut sesuai dengan nama dan NRP mahasiswa
    Nama Mahasiswa &:& Helmika Mahendra Priyanto \\
    Nomor Pokok &:& 07211840000076\\

    % Ubah kalimat berikut sesuai dengan semester pengajuan proposal
    Semester &:& Ganjil 2021/2022 \\

    % Ubah kalimat-kalimat berikut sesuai dengan nama-nama dosen pembimbing
    Dosen Pembimbing &:& 1. Reza Fuad Rachmadi, S.T., M.T., Ph.D. \\
    & & 2. Dr.Supeno Mardi Susiki Nugroho ST., M.T. \\

    % Ubah kalimat berikut sesuai dengan judul tugas akhir
    Judul Tugas Akhir &:& \textbf{Deteksi Helm Keselamatan Kerja menggunakan CNN} \\

    Uraian Tugas Akhir &:& \\
  \end{tabular}
\end{adjustwidth}

% Ubah paragraf berikut sesuai dengan uraian dari tugas akhir
Keselamatan dan Kesehatan Kerja bertujuan meningkatkan standar dan kualitas kerja 
di era modern ini. Pengaplikasiannya pun beragam, salah satunya yaitu penerapan penggunaan 
helm keselamatan kerja atau helm proyek atau Hard Hat. Medan proyek konstruksi yang berisiko 
tinggi menjadi alasan utama pekerja proyek harus benar - benar mematuhi aturan penggunaan APD, 
dimana salah satunya penggunaan helm proyek. Helm proyek membantu mengurangi dampak benturan 
misal saat pengguna terjatuh atau tertimpa benda berat atau tajam. Tetapi tidak semua personel 
lapangan akan dengan sendirinya mematuhi aturan ini sehingga diperlukannya pengawasan dalam 
penerapan penggunaan helm proyek sebagai salah satu APD keselamatan kerja. Pengawas atau 
supervisor lapangan yang dikerahkan adalah personil manusia yang juga memiliki batasannya 
sebagai manusia. Kondisi lapangan proyek yang luas dan banyaknya personil lapangan akan 
menjadi suatu kesulitan untuk pengawas manusia untuk memastikan tiap personil lapangan 
mematuhi aturan penggunaan helm keselamatan kerja. Maka dari itu, dalam penelitian ini diambil 
suatu tujuan yaitu merancang sistem yang dapat mendeteksi penggunaan helm proyek secara otomatis. Dalam perancangan sistem ini, akan memanfaat Convolutional Neural Network yang didesain untuk rekognisi data dua dimensi. Sistem yang sudah jadi akan diuji pada lapangan proyek konstruksi.
\vspace{1ex}

\begin{flushright}
  % Ubah kalimat berikut sesuai dengan tempat, bulan, dan tahun penulisan
  Surabaya, 9 Desember 2021
\end{flushright}
\vspace{1ex}

\begin{center}

  \begin{multicols}{2}

    Dosen Pembimbing 1
    \vspace{12ex}

    % Ubah kalimat-kalimat berikut sesuai dengan nama dan NIP dosen pembimbing pertama
    \underline{[Reza Fuad Rachmadi, S.T., M.T., Ph.D.]} \\
    NIP. 198504032012121000

    \columnbreak

    Dosen Pembimbing 2
    \vspace{12ex}

    % Ubah kalimat-kalimat berikut sesuai dengan nama dan NIP dosen pembimbing kedua
    \underline{[Dr.Supeno Mardi Susiki Nugroho ST., M.T.]} \\
    NIP. 197003131995121001

  \end{multicols}
  \vspace{6ex}

  Mengetahui, \\
  % Ubah kalimat berikut sesuai dengan jabatan kepala departemen
  Kepala Departemen Teknik Komputer FTEIC - ITS
  \vspace{12ex}

  % Ubah kalimat-kalimat berikut sesuai dengan nama dan NIP kepala departemen
  \underline{Dr. Supeno Mardi Susiki Nugroho, S.T., M.T.} \\
  NIP. 197003131995121001

\end{center}